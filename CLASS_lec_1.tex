\documentclass[dvipsnames]{beamer}
\mode<presentation>
\usepackage{graphicx}
\usepackage{wasysym}
\usepackage{hyperref}
\definecolor{links}{HTML}{2A1B81}
\hypersetup{colorlinks,linkcolor=,urlcolor=links}
\usepackage{fancyhdr}
\usepackage{multicol}
\usepackage{multirow}
\usepackage{float}
\usepackage{amssymb}
\usepackage{scalerel,stackengine,amsmath}

\usepackage[normalem]{ulem}
\def\Put(#1,#2)#3{\leavevmode\makebox(0,0){\put(#1,#2){#3}}}

\usepackage{tikz}
\usetikzlibrary{arrows,shapes,chains}
\usetikzlibrary{positioning,decorations.pathreplacing}
\usetikzlibrary{fit,shapes.misc}
\usetikzlibrary{decorations.pathreplacing,calc}
\newcommand{\tikzmark}[1]{\tikz[overlay,remember picture] \node (#1) {};}
\tikzset{
	startstop/.style={
		rectangle, 
		rounded corners,
		minimum width=3cm, 
		minimum height=1cm,
		align=center, 
		draw=black, 
		fill=red!30
	},
	process/.style={
		rectangle, 
		minimum width=3cm, 
		minimum height=1cm, 
		align=center, 
		draw=black, 
		fill=blue!30
	},
	decision/.style={
		rectangle, 
		minimum width=3cm, 
		minimum height=1cm, align=center, 
		draw=black, 
		fill=green!30
	},
	arrow/.style={thick,->,>=stealth},
	dec/.style={
		ellipse, 
		align=center, 
		draw=black, 
		fill=green!30
	},
}
\tikzset{cross/.style={cross out, draw, 
		minimum size=2*(#1-\pgflinewidth), 
		inner sep=0pt, outer sep=0pt,color=red},
	decoration={brace},
	tuborg/.style={decorate}}

\tikzset{
	invisible/.style={opacity=0},
	visible on/.style={alt=#1{}{invisible}},
	alt/.code args={<#1>#2#3}{%
		\alt<#1>{\pgfkeysalso{#2}}{\pgfkeysalso{#3}} % \pgfkeysalso doesn't change the path
	},
}  


\usepackage{geometry}
\usepackage{appendixnumberbeamer}

\usepackage{listliketab} %make itemize env behaving like tables !
\usepackage{listings}
\usepackage{setspace}
\usepackage{color}
% General Colors
\definecolor{deepblue}{rgb}{0,0,0.5}
\definecolor{deepred}{rgb}{0.6,0,0}
\definecolor{deepgreen}{rgb}{0,0.5,0}

% Colors for Python
\definecolor{Code}{rgb}{0,0,0}
\definecolor{Decorators}{rgb}{0.5,0.5,0.5}
\definecolor{Numbers}{rgb}{0.5,0,0}
\definecolor{MatchingBrackets}{rgb}{0.25,0.5,0.5}
\definecolor{Keywords}{rgb}{0,0,0.5}
\definecolor{Strings}{rgb}{0,0.63,0}
\definecolor{Comments}{rgb}{0.4,0.4,0.4}
\definecolor{Backquotes}{rgb}{0,0,0}
\definecolor{Classname}{rgb}{0,0,0}
\definecolor{FunctionName}{rgb}{0,0,0}
\definecolor{Operators}{rgb}{0,0,0}
\definecolor{Background}{rgb}{0.93,0.93,0.93}

\definecolor{RWTHbluedark}{RGB}{0,84,159}
\definecolor{RWTHbluelight}{RGB}{142,186,229}
\definecolor{RWTHmagenta}{RGB}{227,0,102}
\definecolor{RWTHmagentalight}{RGB}{249,210,218}
\definecolor{RWTHyellow}{RGB}{255,237,0}
\definecolor{RWTHorange}{RGB}{246,168,0}
\definecolor{RWTHred}{RGB}{204,7,30}
\definecolor{RWTHgreen}{RGB}{87,171,39}
\definecolor{RWTHlila}{RGB}{122,111,172}
\definecolor{RWTHbordeaux}{RGB}{161,16,53}


\newcommand{\self}{\color{CadetBlue}}


\usepackage{epstopdf}
%\usepackage[urlcolor=magenta]{hyperref}
\usepackage{hyperref}
\usepackage{wasysym}
\hypersetup{urlcolor=magenta}


% Default fixed font does not support bold face
% \DeclareFixedFont{\ttb}{T1}{txtt}{bx}{n}{12} % for bold
% \DeclareFixedFont{\ttm}{T1}{txtt}{m}{n}{12}  % for normal


% Python style for highlighting
\newcommand\pythonstyle{\lstset{
showspaces=false,
showtabs=false,
showstringspaces=false,
tabsize=2,
breaklines=true,
% Basic
%basicstyle=\ttfamily\footnotesize\setstretch{1},
basicstyle=\ttfamily\footnotesize\color{black},
backgroundcolor=\color{Background},
language=Python,
% Comments
commentstyle=\color{Comments}\slshape,
% Strings
stringstyle=\color{Strings},
morecomment=[s][\color{Comments}]{"""}{"""},
morecomment=[s][\color{Strings}]{'''}{'''},
morecomment=[l][\color{BurntOrange}]{\@},
% keywords
keywordstyle={\color{Keywords}\bfseries},
keywordstyle=[2]{\color{Magenta}\bfseries},
keywordstyle=[3]{\color{deepred}\bfseries},
keywords={from,class,def,for,while,if,is,in,elif,else,not,and,or,print,break,continue,return,True,False,None,access,as,del,except,exec,finally,global,lambda,pass,print,raise,try,assert},
% additional keywords
keywords=[2]{import, dir, range},
keywords=[3]{__init__},
emph={self},
emphstyle={\self},
%
}}

% C style for highlighting
\newcommand\cstyle{\lstset{
  %language=C,
  showspaces=false,
  showtabs=false,
  showstringspaces=false,
  tabsize=2,
  basicstyle=\ttfamily\scriptsize\color{black},
  backgroundcolor=\color{Background},
  language=C,
  breaklines=true,
  % Comments
  commentstyle=\color{Comments}\slshape,
  % Strings
  stringstyle=\color{Strings},
  %keywordstyle=\color{Keywords}\ttfamily,
  keywordstyle={\color{Keywords}\bfseries},
  keywordstyle=[2]{\color{Magenta}\bfseries},
  keywordstyle=[3]{\color{deepred}\bfseries},
  keywordstyle=[4]{\color{MidnightBlue}\bfseries},
  keywords={for, if, else, return, break, continue, do, double, float, int, char, enum, struct, long, signed, include},
  keywords=[2]{fprintf, sprintf, printf, scanf, sscanf, fscanf},
  keywords=[3]{class_call, class_test, class_alloc, class_calloc,class_define_index},
  keywords=[4]{stderr, stdout, stdin},
  stringstyle=\color{Strings}\ttfamily,
  commentstyle=\color{Comments}\slshape,
  morecomment=[l][\color{magenta}]{\#},
  morecomment=[s][\color{Strings}]{"}{"},
  morecomment=[s][\color{Strings}]{'}{'},
}}

% C style for highlighting
\newcommand\smallcstyle{\lstset{
		%language=C,
		showspaces=false,
		showtabs=false,
		showstringspaces=false,
		tabsize=2,
		basicstyle=\ttfamily\tiny\color{black},
		backgroundcolor=\color{Background},
		language=C,
		breaklines=true,
		% Comments
		commentstyle=\color{Comments}\slshape,
		% Strings
		stringstyle=\color{Strings},
		%keywordstyle=\color{Keywords}\ttfamily,
		keywordstyle={\color{Keywords}\bfseries},
		keywordstyle=[2]{\color{Magenta}\bfseries},
		keywordstyle=[3]{\color{deepred}\bfseries},
		keywordstyle=[4]{\color{MidnightBlue}\bfseries},
		keywords={for, if, else, return, break, continue, do, double, float, int, char, enum, struct, long, signed, include},
		keywords=[2]{fprintf, sprintf, printf, scanf, sscanf, fscanf},
		keywords=[3]{class_call, class_test, class_alloc, class_calloc},
		keywords=[4]{stderr, stdout, stdin},
		stringstyle=\color{Strings}\ttfamily,
		commentstyle=\color{Comments}\slshape,
		morecomment=[l][\color{magenta}]{\#},
		morecomment=[s][\color{Strings}]{"}{"},
		morecomment=[s][\color{Strings}]{'}{'},
}}

% Python environment
\lstnewenvironment{python}[1][]
{
\pythonstyle
\lstset{#1}
}
{}

% Python for external files
\newcommand\pythonexternal[2]{{
\pythonstyle
\lstinputlisting[#1]{#2}}}

% Python for inline
\newcommand\pythoninline[1]{{\pythonstyle\lstinline!#1!}}

% C new environnement
% Python environment
\lstnewenvironment{class}[1][]
{
\cstyle
\lstset{moredelim=[is][\color{red}]{<@}{@>},#1}
}
{}

\lstnewenvironment{smallclass}[1][]
{
	\smallcstyle
	\lstset{moredelim=[is][\color{red}]{<@}{@>},#1}
}
{}
% C for external files
\newcommand\cexternal[2]{{ 
\cstyle
\lstinputlisting[#1]{#2}}}

\newcommand\cinline[1]{{\cstyle\lstinline[]!#1!}}

\newcommand{\equalhat}{\mathrel{\stackon[1.5pt]{=}{\stretchto{%
				\scalerel*[\widthof{=}]{\wedge}{\rule{1ex}{3ex}}}{0.5ex}}}}

% Personal colors
\newcommand{\mygray}{\only{\color{gray}}}
\newcommand{\mywhite}{\only{\color{white}}}
\newcommand{\myblack}{\only{\color{black}}}
\newcommand{\Blue}{\color{Blue}}

%\newcommand{\Red}{\color{BrickRed}}
\newcommand{\Red}{\color{RWTHred}}
\newcommand{\Green}{\color{PineGreen}}
\newcommand{\Purple}{\color{Mulberry}}
\newcommand{\Grey}{\color{gray}}

\renewcommand\mathfamilydefault{\rmdefault}
\usetheme{Warsaw}
\usecolortheme{whale}

\usepackage[T1]{fontenc}
\usepackage[usefilenames,DefaultFeatures={Ligatures=Common}]{plex-otf} %
\usefonttheme{serif}
\setbeamertemplate{itemize item}[circle]
\setbeamertemplate{itemize subitem}[circle]
\setbeamertemplate{itemize subsubitem}[circle]
\setbeamertemplate{enumerate item}[circle]
\setbeamertemplate{enumerate subitem}[circle]
% \renewcommand{\labelitemi}{$\circ$}

% particular color theme
\setbeamercolor{normal text}{fg=RWTHbluedark}
\setbeamercolor{palette primary}{bg=RWTHmagentalight,fg=black}
\setbeamercolor{palette secondary}{bg=RWTHbordeaux,fg=white}
\setbeamercolor{palette tertiary}{bg=RWTHorange,fg=white}
\setbeamercolor{palette quaternary}{bg=RWTHbordeaux,fg=white}
\setbeamercolor{structure}{fg=RWTHbordeaux} % itemize, enumerate, etc
\setbeamercolor{block title}{bg=RWTHbordeaux,fg=white}

\makeatletter
\renewcommand\verbatim@font{\color{black}\normalfont\ttfamily}
\makeatletter

\title[CLASS Basics\hspace{25mm} \insertframenumber/\inserttotalframenumber]{Cosmological Linear Anisotropy Solving System {\scshape (CLASS)}}

\newcommand{\CLASS}{\texttt{class}}
\newcommand{\classy}{\texttt{classy}}
\newcommand{\location}{Les Karellis}
\newcommand{\ecolefromdate}{17}
\newcommand{\ecoletodate}{30}
\author[\ecolefromdate-\ecoletodate.08.2025 \hspace{15mm} M. Mosbech]{Markus R. Mosbech}



\begin{document}


\begin{frame}

\begin{block}{
\begin{center}\Large CLASS\end{center}}
\begin{center}\small Cosmological Linear Anisotropy Solving System \end{center}
\end{block}

\scriptsize

\begin{center}
	\includegraphics[width=5cm,angle=0]{Figures/Logo1b_blue.pdf}\\
	%\framebox{
	Markus Mosbech\\
	Institute for Theoretical Particle Physics and Cosmology, RWTH Aachen University\\
	\mbox{}\\
	\mbox{}\\
	\location, France, \ecolefromdate{}-\ecoletodate{} Aug 2025
	%}
	\vfill
	These slides available at \url{https://github.com/MarkMos/class_lecture}\\
	Visit \url{http://class-code.net/} for more info!
\end{center}

\end{frame}


\scriptsize

\begin{frame}[fragile]
\frametitle{{\Red \CLASS{}} in \location}

\mbox{}\\\mbox{}\\
What to expect in this first lecture:
\vspace*{0.5\baselineskip}\mbox{}
\bgroup 
\def\arraystretch{1.15}
\begin{tabular}{lll}
$\bullet$&Basics:& Why use {\Red \CLASS{}}?\\
$\bullet$&Usage:& Installation\\
% $\bullet$&Usage:& Terminal\\
% $\bullet$&Usage:& Plotting\\
$\bullet$&Usage:& Python Interface \\
$\bullet$&Usage:& Samplers \\
$\bullet$&Basics:& Existing Species \\
$\bullet$&Basics:& Module Overview \\
\end{tabular}
\egroup

\mbox{}\\
We will learn {\Red how to use \CLASS{}} and {\Red which models} can be run with it.\\\mbox{}\\
% \begin{center}\includegraphics[width=8cm,angle=0]{logo-ecole-300.png}\end{center}

\end{frame}

\begin{frame}[fragile]
	\frametitle{What is an Einstein-Boltzmann solver?}

	Often just called a \emph{Boltzmann code} for brevity, a typical Boltzmann code will:
	\vspace{0.5\baselineskip}
	\begin{itemize}
		\item Solve coupled Einstein and Boltzmann equations.\\
		\item Generally work at linear level in perturbation theory. \\
		\item Compute global (Background+Themodynamic) quantities \emph{and} perturbations.
	\end{itemize}

	% \mbox{
	\begin{equation}
		\underbrace{G_{\mu \nu} = 8 \pi T_{\mu \nu}}_{\text{Einstein-equation}} \qquad \qquad \underbrace{\frac{\mathrm{d} f}{\mathrm{d} \lambda} = C[f]}_{\text{Boltzmann-equation}}
	\end{equation}

\end{frame}

\begin{frame}[fragile]
	\frametitle{Why use a Boltzmann code?}
	Modern Boltzmann codes offer:
	\vspace{0.5\baselineskip}
	\begin{itemize}
		\item History of the universe at the global level ($H(z)$, $\rho_i(z)$, etc.) \pause
		\item Thermal history of the universe ($T_b(z)$, $x_e$, $\tau$, etc.) \pause
		\item Evolution of (linear) perturbations ($\delta_i$, $\theta_i$, $\psi$, $\phi$, etc.) \pause
		\item Fourier space transfer functions ($T(k)$) \pause
		\item CMB spectra, both lensed and unlensed ($C_\ell^{TT}$,$C_\ell^{TE}$,$C_\ell^{EE}$,$C_\ell^{BB}$) \pause
		\item Linear matter power spectrum, galaxy counts, cosmic shear ($\xi^\pm$,$C_\ell^{dd}$,$P_\mathrm{lin}(k)$) \pause
		\item Emulated non-linear power spectra \pause
		\item CMB spectral distortions \pause
	\end{itemize}
	All computed in a matter of seconds!

\end{frame}

\begin{frame}[fragile]
	\frametitle{Why use a Boltzmann code?}
	This has several use cases:
	\vspace{0.5\baselineskip}
	\begin{itemize}
		\item Analysis of CMB experiments
		\item Analysis of LSS experiments
		\item Initial conditions for non-linear simulations ($N$-body, etc.)
		\item Consistent treatment of background/thermodynamic evolution
	\end{itemize}
	All easy to to with {\Red \CLASS{}}!\\
		\vspace{\baselineskip}
	Fast execution $\Rightarrow$ ideal for use in an MCMC pipeline.

\end{frame}

\begin{frame}[fragile]
	\frametitle{Why use \CLASS{}?}
	\CLASS{} is:
	\vspace{0.5\baselineskip}
	\begin{itemize}
		\item Accurate: \CLASS{} \& \texttt{camb} cross-check each other
		\item Versatile: Interfaces with \texttt{MontePython}, \texttt{Cobaya}, \texttt{Cosmosis}, \texttt{Procoli}, \texttt{CosmoPower}, \texttt{OL\'E}, \texttt{CONNECT}, and others!
		\item Comprehensive: Computes a wide range of cosmological observables for a large selection of models beyond $\Lambda$CDM.
		\item Modular and well-documented: ReadTheDocs page and Doxygen documentation, thoroughly commented source code, easy to modify
	\end{itemize}
	All strong arguments to use {\Red \CLASS{}}!\\

\end{frame}

\begin{frame}[fragile]
	\frametitle{Installing \CLASS{}}

	% \begin{columns}
    % \begin{column}{0.5\linewidth}
	\begin{minipage}[t][0.8\textheight][t]{.45\textwidth}
      \begin{block}{Using \CLASS{}}
		\begin{minipage}[t][0.42\textheight][t]{\textwidth}
        If you have no intention of modifying source code:
		\begin{class}
> pip install classy
		\end{class}
		And the \CLASS{} wrapper will be ready to use in your Python environment.\\
		\\
		This is the easiest way to install.
		\end{minipage}
      \end{block}
	  \end{minipage}
	  \hfill
    % \end{column}
    % \begin{column}{0.5\linewidth}
	\begin{minipage}[t][0.8\textheight][t]{.45\textwidth}
      \begin{block}{Modifying \CLASS{}}
		\begin{minipage}[t][0.42\textheight][t]{\textwidth}
		If you wish to modify source code:
        \begin{class}
> git clone git@github.com:lesgourg/class_public.git class
> cd class/
> make clean; make -j
		\end{class}
	  The wrapper can be used in your Python environment, and the binary executable can be called from the terminal.
	  \end{minipage}
      \end{block}
	  \end{minipage}
%     \end{column}
%   \end{columns}

\end{frame}

\begin{frame}[fragile]
	\frametitle{Documentation}
\begin{enumerate}
	\item Basic information and links:\pause
	\begin{itemize}
		\item
		{\scriptsize in the historical {\tt \Red class} webpage \url{http://class-code.net}}\pause
		\item
		{\scriptsize in the pdf manual in \cinline{doc/manual/CLASS\_MANUAL.pdf}} \pause
		\item {\scriptsize the online documentation page (from the previous page, or from \url{https://github.com/lesgourg/class_public/wiki}, click on the link {\tt \Red{online html documentation}})}
					%\begin{center}
					%\includegraphics[height=6cm,angle=0]{figs/rtfm.png}
					%\end{center}
		\item {\scriptsize First three subsections: }
		\begin{itemize}\item {\scriptsize Installation instructions}
			\item {\scriptsize References to many papers for the physics}
			\item {\scriptsize General overview (architecture, input/output, general principles)}
		\end{itemize}
	\end{itemize}
	\pause
	\item More advanced:
	\begin{itemize}
		\item {\scriptsize Old course notes from previous years on \url{https://schoeneberg.github.io/} under ``Resources''}
		\item {\scriptsize several detailed courses on Julien's course webpage \url{https://lesgourg.github.io/courses.html}, especially the courses from Tokyo and NYC}
		\item {\scriptsize Full auto-generated documentation with dependence tree.}
	\end{itemize}
\end{enumerate}
\end{frame}

\begin{frame}[fragile]
	\frametitle{The code structure}


\end{frame}


%\begin{frame}[fragile]
%\frametitle{Context}
%{\tt \Red class} is the 5th public Einstein-Boltzmann solver covering all basic cosmology:
%\begin{enumerate}
%\item
%{\Red COSMICS} package in f77 (Bertschinger 1995)\\
%Basic equations, brute-force $C_l^{TT}$
%%\pause
%\item
%{\Red CMBFAST} in f77 (Seljak \& Zaldarriaga 1996)\\
%Line-of-sight, $C_l^{EE,TE, BB}$, open universe, CMB lensing
%%\pause
%\item
%{\Red CAMB} in f90/2000 (Lewis \& Challinor 1999)\\
%closed universe, better lensing, new algorithms, new approximations, new species, new observables...
%(\url{http://camb.info})
%%\pause
%\item
%{\Red CMBEASY} in C++ (Doran 2003)
%%\pause
%\item
%{\tt \Red class} in C (Lesgourgues \& Tram 2011)\\
%simpler polarisation equations, new algorithms, new approximations, new species, new observables...
%(\url{http://class-code.net})
%\end{enumerate}
%... and there might still be 1 or 2 more!
%But only {\Red CAMB} and {\tt \Red class} are currently developed and kept to high precision level.
%\end{frame}

%
%\begin{frame}[fragile]
%\frametitle{Module structure of {\tt \Red class}}
%
%The \cinline{main(command-line-args)} function of CLASS located in \cinline{main/class.c} could only contain:
%\begin{class}
%int main(args) {
%  input_init(args,ppr,pba,pth,ppt,ptr,ppm,pha,pfo,ple,pop);
%  background_init(ppr,pba);
%  thermodynamics_init(ppr,pba,pth); 
%  perturbations_init(ppr,pba,pth,ppt);
%  primordial_init(ppr,ppt,ppm);
%  fourier_init(ppr,pba,pth,ppt,ppm,pfo);
%  transfer_init(ppr,pba,pth,ppt,pfo,ptr);
%  harmonic_init(ppr,pba,ppt,ppm,pfo,ptr,pha);
%  lensing_init(ppr,ppt,pha,pfo,ple);
%  distortions_init(ppr,pba,pth,ppt,pha,pfo,psd);
%  output_init(pba,pth,ppt,ppm,ptr,pha,pfo,ple,pop);
%  /* all calculations done, free the structures */
%  lensing_free(ple);
%  distortions_free(psd);
%  harmonic_free(pha);
%  transfer_free(ptr);
%  fourier_free(pfo);
%  primordial_free(ppm);
%  perturbations_free(ppt);
%  thermodynamics_free(pth);
%  background_free(pba);
%}
%\end{class}
%Of course we also need error management!
%\end{frame}







\end{document}

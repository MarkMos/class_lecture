\documentclass[dvipsnames]{beamer}
\mode<presentation>
\usepackage{graphicx}
\usepackage{wasysym}
\usepackage{hyperref}
% \definecolor{links}{HTML}{2A1B81}
\definecolor{links}{RGB}{122,111,172}
\hypersetup{colorlinks,linkcolor=,urlcolor=links}
\usepackage{fancyhdr}
\usepackage{multicol}
\usepackage{multirow}
\usepackage{float}
\usepackage{amssymb}
\usepackage{scalerel,stackengine,amsmath}
\usepackage{makecell}

\usepackage[normalem]{ulem}
\def\Put(#1,#2)#3{\leavevmode\makebox(0,0){\put(#1,#2){#3}}}

\usepackage{tikz}
\usetikzlibrary{arrows,shapes,chains}
\usetikzlibrary{positioning,decorations.pathreplacing}
\usetikzlibrary{fit,shapes.misc}
\usetikzlibrary{decorations.pathreplacing,calc}
\newcommand{\tikzmark}[1]{\tikz[overlay,remember picture] \node (#1) {};}
\tikzset{
	startstop/.style={
		rectangle, 
		rounded corners,
		minimum width=3cm, 
		minimum height=1cm,
		align=center, 
		draw=black, 
		fill=RWTHbluelight,
		text=black
	},
	process/.style={
		rectangle, 
		minimum width=3cm, 
		minimum height=1cm, 
		align=center, 
		draw=black, 
		fill=RWTHmagentalight,
		text=black
	},
	decision/.style={
		rectangle, 
		minimum width=3cm, 
		minimum height=1cm, align=center, 
		draw=black, 
		fill=green!30
	},
	arrow/.style={thick,->,>=stealth},
	dec/.style={
		ellipse, 
		align=center, 
		draw=black, 
		fill=green!30
	},
}
\tikzset{cross/.style={cross out, draw, 
		minimum size=2*(#1-\pgflinewidth), 
		inner sep=0pt, outer sep=0pt,color=red},
	decoration={brace},
	tuborg/.style={decorate}}

\tikzset{
	invisible/.style={opacity=0},
	visible on/.style={alt=#1{}{invisible}},
	alt/.code args={<#1>#2#3}{%
		\alt<#1>{\pgfkeysalso{#2}}{\pgfkeysalso{#3}} % \pgfkeysalso doesn't change the path
	},
}  


\usepackage{geometry}
\usepackage{appendixnumberbeamer}

\usepackage{listliketab} %make itemize env behaving like tables !
\usepackage{listings}
\usepackage{setspace}
\usepackage{color}
% General Colors
\definecolor{deepblue}{rgb}{0,0,0.5}
\definecolor{deepred}{rgb}{0.6,0,0}
\definecolor{deepgreen}{rgb}{0,0.5,0}

% Colors for Python
\definecolor{Code}{rgb}{0,0,0}
\definecolor{Decorators}{rgb}{0.5,0.5,0.5}
\definecolor{Numbers}{rgb}{0.5,0,0}
\definecolor{MatchingBrackets}{rgb}{0.25,0.5,0.5}
\definecolor{Keywords}{rgb}{0,0,0.5}
\definecolor{Strings}{rgb}{0,0.63,0}
% \definecolor{Comments}{rgb}{0.4,0.4,0.4}
\definecolor{Comments}{rgb}{0,0.4,0}
\definecolor{Backquotes}{rgb}{0,0,0}
\definecolor{Classname}{rgb}{0,0,0}
\definecolor{FunctionName}{rgb}{0,0,0}
\definecolor{Operators}{rgb}{0,0,0}
\definecolor{Background}{rgb}{0.93,0.93,0.93}

\definecolor{MyDarkGray}{RGB}{40,40,40}

\definecolor{RWTHbluedark}{RGB}{0,84,159}
\definecolor{RWTHbluelight}{RGB}{142,186,229}
\definecolor{RWTHmagenta}{RGB}{227,0,102}
\definecolor{RWTHmagentalight}{RGB}{249,210,218}
\definecolor{RWTHyellow}{RGB}{255,237,0}
\definecolor{RWTHorange}{RGB}{246,168,0}
\definecolor{RWTHred}{RGB}{204,7,30}
\definecolor{RWTHgreen}{RGB}{87,171,39}
\definecolor{RWTHlila}{RGB}{122,111,172}
\definecolor{RWTHbordeaux}{RGB}{161,16,53}


\newcommand{\self}{\color{CadetBlue}}


\usepackage{epstopdf}
%\usepackage[urlcolor=magenta]{hyperref}
\usepackage{hyperref}
\usepackage{wasysym}
\hypersetup{urlcolor=links}


% Default fixed font does not support bold face
% \DeclareFixedFont{\ttb}{T1}{txtt}{bx}{n}{12} % for bold
% \DeclareFixedFont{\ttm}{T1}{txtt}{m}{n}{12}  % for normal


% Python style for highlighting
\newcommand\pythonstyle{\lstset{
showspaces=false,
showtabs=false,
showstringspaces=false,
tabsize=2,
breaklines=true,
% Basic
%basicstyle=\ttfamily\footnotesize\setstretch{1},
basicstyle=\ttfamily\footnotesize\color{black},
backgroundcolor=\color{Background},
language=Python,
% Comments
commentstyle=\color{Comments}\slshape,
% Strings
stringstyle=\color{Strings},
morecomment=[s][\color{Comments}]{"""}{"""},
morecomment=[s][\color{Strings}]{'''}{'''},
morecomment=[l][\color{BurntOrange}]{\@},
% keywords
keywordstyle={\color{Keywords}\bfseries},
keywordstyle=[2]{\color{Magenta}\bfseries},
keywordstyle=[3]{\color{deepred}\bfseries},
keywords={from,class,def,for,while,if,is,in,elif,else,not,and,or,print,break,continue,return,True,False,None,access,as,del,except,exec,finally,global,lambda,pass,print,raise,try,assert},
% additional keywords
keywords=[2]{import, dir, range},
keywords=[3]{__init__},
emph={self},
emphstyle={\self},
%
}}

% C style for highlighting
\newcommand\cstyle{\lstset{
  %language=C,
  showspaces=false,
  showtabs=false,
  showstringspaces=false,
  tabsize=2,
  basicstyle=\ttfamily\scriptsize\color{black},
  backgroundcolor=\color{Background},
  language=C,
  breaklines=true,
  % Comments
  commentstyle=\color{Comments}\slshape,
  % Strings
  stringstyle=\color{Strings},
  %keywordstyle=\color{Keywords}\ttfamily,
  keywordstyle={\color{Keywords}\bfseries},
  keywordstyle=[2]{\color{Magenta}\bfseries},
  keywordstyle=[3]{\color{deepred}\bfseries},
  keywordstyle=[4]{\color{MidnightBlue}\bfseries},
  keywords={for, if, else, return, break, continue, do, double, float, int, char, enum, struct, long, signed, include},
  keywords=[2]{fprintf, sprintf, printf, scanf, sscanf, fscanf},
  keywords=[3]{class_call, class_test, class_alloc, class_calloc,class_define_index},
  keywords=[4]{stderr, stdout, stdin},
  stringstyle=\color{Strings}\ttfamily,
  commentstyle=\color{Comments}\slshape,
  morecomment=[l][\color{Comments}]{\#},
  morecomment=[s][\color{Strings}]{"}{"},
  morecomment=[s][\color{Strings}]{'}{'},
}}

% C style for highlighting
\newcommand\smallcstyle{\lstset{
		%language=C,
		showspaces=false,
		showtabs=false,
		showstringspaces=false,
		tabsize=2,
		basicstyle=\ttfamily\tiny\color{black},
		backgroundcolor=\color{Background},
		language=C,
		breaklines=true,
		% Comments
		commentstyle=\color{Comments}\slshape,
		% Strings
		stringstyle=\color{Strings},
		%keywordstyle=\color{Keywords}\ttfamily,
		keywordstyle={\color{Keywords}\bfseries},
		keywordstyle=[2]{\color{Magenta}\bfseries},
		keywordstyle=[3]{\color{deepred}\bfseries},
		keywordstyle=[4]{\color{MidnightBlue}\bfseries},
		keywords={for, if, else, return, break, continue, do, double, float, int, char, enum, struct, long, signed, include},
		keywords=[2]{fprintf, sprintf, printf, scanf, sscanf, fscanf},
		keywords=[3]{class_call, class_test, class_alloc, class_calloc},
		keywords=[4]{stderr, stdout, stdin},
		stringstyle=\color{Strings}\ttfamily,
		commentstyle=\color{Comments}\slshape,
		morecomment=[l][\color{Comments}]{\#},
		morecomment=[s][\color{Strings}]{"}{"},
		morecomment=[s][\color{Strings}]{'}{'},
}}

% Python environment
\lstnewenvironment{python}[1][]
{
\pythonstyle
\lstset{#1}
}
{}

% Python for external files
\newcommand\pythonexternal[2]{{
\pythonstyle
\lstinputlisting[#1]{#2}}}

% Python for inline
\newcommand\pythoninline[1]{{\pythonstyle\lstinline!#1!}}

% C new environnement
% Python environment
\lstnewenvironment{class}[1][]
{
\cstyle
\lstset{moredelim=[is][\color{red}]{<@}{@>},#1}
}
{}

\lstnewenvironment{smallclass}[1][]
{
	\smallcstyle
	\lstset{moredelim=[is][\color{red}]{<@}{@>},#1}
}
{}
% C for external files
\newcommand\cexternal[2]{{ 
\cstyle
\lstinputlisting[#1]{#2}}}

\newcommand\cinline[1]{{\cstyle\lstinline[]!#1!}}

\newcommand{\equalhat}{\mathrel{\stackon[1.5pt]{=}{\stretchto{%
				\scalerel*[\widthof{=}]{\wedge}{\rule{1ex}{3ex}}}{0.5ex}}}}

% Personal colors
\newcommand{\mygray}{\only{\color{gray}}}
\newcommand{\mywhite}{\only{\color{white}}}
\newcommand{\myblack}{\only{\color{black}}}
\newcommand{\Blue}{\color{Blue}}

%\newcommand{\Red}{\color{BrickRed}}
\newcommand{\Red}{\color{RWTHred}}
\newcommand{\Green}{\color{PineGreen}}
\newcommand{\Purple}{\color{Mulberry}}
\newcommand{\Grey}{\color{gray}}

\renewcommand\mathfamilydefault{\rmdefault}
\usetheme{Warsaw}
\usecolortheme{whale}

\usepackage[T1]{fontenc}
\usepackage[usefilenames,DefaultFeatures={Ligatures=Common}]{plex-otf} %
\usefonttheme{serif}
\setbeamertemplate{itemize item}[circle]
\setbeamertemplate{itemize subitem}[circle]
\setbeamertemplate{itemize subsubitem}[circle]
\setbeamertemplate{enumerate item}[circle]
\setbeamertemplate{enumerate subitem}[circle]
% \renewcommand{\labelitemi}{$\circ$}

% particular color theme
\setbeamercolor{normal text}{fg=MyDarkGray}
% \setbeamercolor{normal text}{fg=RWTHbluedark}
\setbeamercolor{palette primary}{bg=RWTHmagentalight,fg=black}
\setbeamercolor{palette secondary}{bg=RWTHbordeaux,fg=white}
\setbeamercolor{palette tertiary}{bg=RWTHorange,fg=white}
\setbeamercolor{palette quaternary}{bg=RWTHbordeaux,fg=white}
\setbeamercolor{structure}{fg=RWTHbordeaux} % itemize, enumerate, etc
\setbeamercolor{block title}{bg=RWTHbordeaux,fg=white}

\makeatletter
\renewcommand\verbatim@font{\color{black}\normalfont\ttfamily}
\makeatletter

\title[CLASS Basics\hspace{25mm} \insertframenumber/\inserttotalframenumber]{Cosmological Linear Anisotropy Solving System {\scshape (CLASS)}}

\newcommand{\CLASS}{\texttt{class}}
\newcommand{\classy}{\texttt{classy}}
\newcommand{\location}{Les Karellis}
\newcommand{\ecolefromdate}{17}
\newcommand{\ecoletodate}{30}
\author[\ecolefromdate-\ecoletodate.08.2025 \hspace{15mm} M. Mosbech]{Markus R. Mosbech}
